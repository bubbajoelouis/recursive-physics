\documentclass{article}
\usepackage{amsmath, amssymb, geometry}
\geometry{a4paper, margin=1in}

\title{Recursive Surface Interactions and Dimensional Scaling in Energy Systems}
\author{Unified Theory of Energy Framework}
\date{\today}

\begin{document}

\maketitle

\begin{abstract}
This paper explores the Degrees of Surface Interaction (D) and their recursive relationship to energy exchange across different Mass Structures. We introduce a dimensional scaling model wherein celestial bodies, materials, and nuclear structures can be classified based on their equilibrium between Radiation, Gravitation, and Particulate Motion. The classification accounts for fractalized transitions in dimensionality and the role of recursive energy storage. This framework enables a reevaluation of stellar structures, planetary bodies, and metallurgical failure events.
\end{abstract}

\section{Introduction}
The Unified Theory of Energy (UTE) posits that energy exists in three distinct states: as Radiation, as Gravitation, and as Particulate Motion. The Degrees of Surface Interaction (D) describe how these energy states interact at different structural levels. Unlike classical physics, which assumes fixed-dimensional frameworks, we explore how energy recursively transitions across dimensions.

\section{Mathematical Basis: Recursive Surface Interactions}
Each energy state follows a recursive hierarchy wherein:
\begin{equation}
    G = \frac{E}{R}, \quad R = \frac{E}{G}, \quad G/R \neq \infty, \quad R/G \neq \infty.
\end{equation}
The limitation of these ratios ensures that energy states remain in equilibrium within their respective Radiation Coordinate Systems.

\section{Degrees of Surface Interaction Across Physical Systems}

\subsection{Earth's Surface ($D=2$)}
\begin{itemize}
    \item The Earth's surface is defined by energy exchange in a two-dimensional interaction layer.
    \item Particulate Motion (weather, ocean currents) mediates Radiation and Gravitation.
    \item Life emerges at $D=4$, where recursive biochemical interactions create self-organizing systems.
\end{itemize}

\subsection{Sun's Surface ($D=3$)}
\begin{itemize}
    \item The Sun's photosphere represents a three-dimensional energy transformation layer.
    \item Stored Gravitation maintains the structural integrity of the plasma.
    \item Radiation extends outward, preventing collapse and driving stellar expansion.
\end{itemize}

\subsection{Oxygen Nucleus ($D=3$)}
\begin{itemize}
    \item The nucleus of an oxygen atom contains stored Gravitation (binding energy).
    \item Particulate Motion stabilizes the proton-neutron interactions.
    \item Radiation is released in nuclear reactions, following recursive energy scaling.
\end{itemize}

\subsection{Moon's Surface ($D=1.5 - 2$)}
\begin{itemize}
    \item The Moon lacks an active hydrosphere but does contain minimal gaseous interactions.
    \item Without an Earth-like atmosphere, its energy exchange is distinct from planets with biospheres.
    \item Surface Interactions likely involve slow-scale radiation equilibrium rather than active chemical cycling.
\end{itemize}

\subsection{Steel and Fracture Mechanics ($D=2.9$ to $D=3$)}
\begin{itemize}
    \item Steel, as a lattice, undergoes recursive Surface Interactions.
    \item When fractured, the transition from $D=3$ to $D=2.9$ reflects increased instability.
    \item A nuclear meltdown represents an extended $D=2.9$ scenario where stored Gravitation rapidly converts to Radiation.
\end{itemize}

\section{Conclusion}
This paper establishes a recursive model of Surface Interactions and energy scaling, demonstrating how different physical systems adhere to varying Degrees of Surface Interaction. The fractalized nature of these interactions suggests a universal scaling principle applicable to celestial mechanics, material science, and nuclear physics. Future research should focus on refining these classifications through empirical measurements of energy exchange in astrophysical and condensed matter systems.

\end{document}
